The emerging field of quantum technologies has many promising applications, one of them is quantum computation (QC), which currently is a very active area of research. Quantum computers make use of some of the quantum mechanical concepts such as superposition, entanglement, and interference to design powerful algorithms. These algorithms could be used to solve some hard problems which would not be classically possible, such as efficient prime-number factoring \cite{shor}, it can also be used to reduce the time complexity of some commonly used algorithms \cite{Grover}. Current quantum computers are very susceptible to decoherence and noise, and thus will not have any commercial use soon, but stand as an important proof of concept. 
The most common model for quantum computation is the circuit model, which is analogous to the classical circuits used for classical computers. Gates are replaced by unitary transformations and bits by qubits. To achieve the computational advantage it is important to construct robust, noise-resilient quantum gates. A good candidate for this is holonomic quantum computation \cite{HQC,NHQC} which is based on the Berry phase \cite{berry} and its non-abelian and/or non-adiabatic generalizations\cite{anandan1,anandan2,zee}. These methods are only dependent on the geometry of the system and thus resilient to local errors in the dynamical evolution.


The idea that our elements of computation should be limited to two-dimensional (qu)bits is sort of an arbitrary choice, it most likely rose out of convenience due to binary logic. So why binary logic? It is simply the easiest non-trivial example, in binary things can be either $1$ or $0$, {\tt True} or {\tt False}, \textbf{on} or \textbf{off}. Due to its simplicity, it's no wonder why this is how the first computer was designed. But are we limited to (qu)bits? As early as 1840 a mechanical trenary (three-valued logic) calculation device was built by Thomas Fowler \cite{tricalc}, and in 1958 the first electronic trenary computer was built by the Soviet Union \cite{setun}. Even though it had many advantages over the binary computer it never saw the same widespread success. There is nothing in theory that forbids a higher dimensional computational basis, even more so when it comes to quantum computers, where the implementation of the elements of computation already surpasses the simplicity of \textbf{on} and \textbf{off}. There are already promising qudit results that show potential \cite{qutrit1,qudit2,qudit3}, and in the review article \cite{qudit} a good overview of the field is given and further research into the topic is encouraged.

In this report we will show how to find a new scheme to implement qudits which could be more efficient than some current ones by making use of dark paths for increased parameter control and auxilary states for increased fidelity. We do this by generalizing the idea of the scheme from \cite{darkpath}. The report is structured as follows; the background/theory section is split into two parts where the first part serves as a quick introduction to the most important aspects of quantum mechanics and as well as the commonly used notation. Then follows a part more concerned with quantum computation, quantum information, and some of the more advanced quantum mechanical concepts that those are built upon.
The two main sections contain the idea of this project, first and explicit example of how a qutrit can be constructed in the Qutrit section and then how the scheme would generalize up to any arbitrary dimension $\geq 3$. The report ends with Conclusions and a brief outlook on what could be built on further upon from this project.


