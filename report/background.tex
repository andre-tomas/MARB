The Background consists of a quick introduction to the most important quantum mechanical concepts and notation. Readers familiar with quantum mechanics may skip it. The second part explores the fundamentals of quantum computation and 

\subsection{Basic quantum mechanics, part I}
This part offers a quick introduction to the necessary quantum mechanics for readers whom are not familiar with the subject. Quantum mechanics is nothing more than linear algebra with fancy notation and some additional rules. The section contains nothing relevant for later parts of the thesis and can be safely skipped. For a more complete introduction I suggest chapter 1 of Sakurai.


\subsubsection{Quantum states and Dirac notation}
First lets define what is meant by the term \textbf{state}. In classical physics a state would be given by the position and momentum of all its individual constituents. An example would be a system of $N$ particles, the state would be given by $\{ (\vec{x}, \vec{p})_i \}_{i=1}^N$, where $\vec{x},\vec{p} \in \mathbb{R}^3$ are the position and momentum in 3 dimensions. In Quantum Mechanics (QM) it more subtle than this, since exact information of the system can not be obtained in the same way. So a state is represented by a normalized vector in a complex vector space. The vector space in which the state vector lives is called a \textbf{Hilbert space} and has the following properties. given two vectors $u,v$ in $H$ they satisfy:

\noindent The inner product is conjugate symmetric
\begin{enumerate}
\item $\inp{u}{v} = \overline{\inp{v}{u}} \in \mathbb{C}$
\end{enumerate}
The inner product is linear in the first argument, for constants $a,b\in \mathbb{C}$
\begin{enumerate}[resume]
\item $\inp{au_1 + bu_2}{v} = a\inp{u_1}{v} + b\inp{u_2}{v}$
\end{enumerate}
The inner product is positive definite 
\begin{enumerate}[resume]
\item $\inp{u}{u} = 0 \iff u = 0$
\end{enumerate}
These properties can be combined to find some other useful facts that holds,
combining the 1st and 2nd property,

\begin{equation}
\inp{v}{au_1 + bu_2} = \overline{\inp{au_1 + bu_2}{v}} = a^{*}\overline{\inp{u_1}{v}} + b^{*}\overline{\inp{u_2}{v}} =   a^{*}\inp{v}{u_1} + b^{*}\inp{v}{u_2}
\end{equation} 
the inner product is anti-linear in the second term.
Using the 1st and 3rd property 
\begin{equation}
\inp{u}{u} = \overline{\inp{u}{u}} \implies \text{Im}(\inp{u}{u}) = 0
\end{equation}
or in words, the inner product of two identical vectors is a real number.

So now we have established that a quantum state is a normalized vector $v$ in a Hilbert space $H$. Now a property of vector spaces is that any vector can be multiplied by a matrix, and the resulting vector will be a new vector in the same vector space. This is what is meant when a quantum state is \textbf{acted} upon. For a matrix $A$ we have that 
\begin{equation}
H \ni v  \xrightarrow{A} Av = v' \in H
\end{equation}
acting on a state alters it in various ways.

Now lets go from this linear algebra notation to the Dirac notation commonly used in quantum mechanics, also know as bra-ket notation.

Vectors are replaced by \textbf{kets},$\ket{}$, or \textbf{bras},$\bra{}$. \\
$v \mapsto \ket{\psi}$ \\
$v^\dagger \mapsto \bra{\psi}$\\
and matrices are replaced by operators\\
$A \mapsto \hat{A}$\\
The same rules applies to these as for the usual vectors.
The label inside the brackets does no in itself have any meanings and are in some sense only just that, labels, but more often than not it is used to represent some property of the state. With this notation the inner product between to states $\ket{\psi},\ket{\varphi}$ is written as 
\begin{equation}
\bra{\psi}\ket{\varphi} = a_1^{*}b_1 + a_2^{*}b_2 + \dots + a_n^{*}b_n = \begin{pmatrix} a_1^{*} & a_2^{*} & \dots & a_n^{*}\end{pmatrix} \begin{pmatrix} b_1 \\ b_2 \\ \vdots \\ b_n\end{pmatrix}
\end{equation}
from the properties defined earlier the relations $(\ket{\psi})^\dagger = \bra{\psi}$ and $ (\bra{\psi}\ket{\varphi})^\dagger = \bra{\varphi}\ket{\psi}$,
this suggest that another way to define the states would simply be
\begin{equation}
\ket{\psi} = \begin{pmatrix}
a_1 \\ a_2 \\ \vdots \\ a_n
\end{pmatrix},\;
\bra{\psi} = (\ket{\psi})^{\dagger} = (a_1^{*}, a_2^{*}, \dots, a_n^{*}),\; a_1,a_2,\dots,a_n \in \mathbb{C}
\end{equation}
which is nothing more than a complex vector.


\subsubsection{Kets, Bras and Operators}
When talking about Quantum mechanics the term \textbf{quantum state}, or more likley just state, is mentioned a lot. A quantum mechanical state is represented by a \textbf{ket}, an complex-valued vector with either finite or infinite entries. Closely related to the ket is the \textbf{bra}, which is the corresponding vector to the ket in the dual space, or more simply, the hermitian conjugate of the ket, see Equation \ref{eq:ket}.
\begin{equation}
\label{eq:ket}
\ket{\psi} = \begin{pmatrix}
a_1 \\ a_2 \\ \vdots \\ a_n
\end{pmatrix},\;
\bra{\psi} = (\ket{\psi})^{\dagger} = (a_1^{*}, a_2^{*}, \dots, a_n^{*}),\; a_1,a_2,\dots,a_n \in \mathbb{C}
\end{equation} 
An arbitrary ket can be rewritten as a linear combination of its eigenkets that span the same space, $\ket{\psi} = \sum_i a_i\ket{i}$ The coefficient is know as the probability amplitude, and $|a_i|^2$ corresponds to the probability to find the state in the $i$th state. 
The inner product of two quantum states is simply written as 
$\bra{\varphi}\ket{\psi} = (\bra{\varphi}\ket{\psi})^\dagger \in \mathbb{C}$. 
An operator can act on a quantum state and corresponds to multiplication with a matrix, and will always yield a new state, ket. 
\begin{equation}
\hat{O}\ket{\psi} = \ket{\psi'}
\end{equation}
Operators are often written in terms of of kets and bras, for example and operator which takes the state $\ket{1}$ and returns the state $\ket{2}$ is written as
\begin{equation}
(\ket{2}\bra{1})\ket{1} = \ket{2}\bra{1}\ket{1} = \ket{2}(\bra{1}\ket{1}) = (\bra{1}\ket{1})\ket{2} = 1 \ket{2} = \ket{2}
\end{equation}
or a identity $2\times 2$ matrix would be
\begin{equation}
 \begin{pmatrix}
 1 & 0 \\ 0 & 1
 \end{pmatrix} = \ket{1}\bra{1} + \ket{2}\bra{2} 
\end{equation}
and so on. An operator for a physical measurable quantity is called an observable.

\subsubsection{Measurements and Observables and Uncertainty}
A quantum mechanical measurement ''breaks'' the superposition of a quantum states and shifts it into a eigenstate of the observable,
\begin{equation}
\ket{\psi} \longrightarrow \ket{i}
\end{equation}
the probability to find the $i$th eigenstate is $|a_i|^2$. 
In quantum mechanics the relation $AB - BA = 0$,where $A$ and $B$ are operators, does not generally hold. This is due to the uncommutative nature of quantum mechanics, it relates to uncertainty but it also required since it a can realize a more complex mathematical structure. The fact that matrices are non-commutative are not surprising, and since operators can be represented by matrices this should not be that confusing. 
To make it more concrete to which degree two operators commute one can define  the commutator on operators $A,B$ as $[A,B] = AB - BA$, which is zero for commuting operators and non-zero otherwise. 
Observables which don't commute are called incompatible observables, a well know pair of incompatible observables are position and momentum, $\mathbf{x}$ and $\mathbf{p}$, which can not be measured to arbitrary precision, this is due to the fact that $[\mathbf{x},\mathbf{p}] \neq 0$. So for two incompatible observables the general uncertainty the measurements will be limited by \note{uncertainty relation here.} 

\subsubsection{Time evolution and the Schrödinger equation}

\subsection{Quantum Computation and Quantum Information theory, part 2}

\subsubsection{The Qubit}
A classical bit is a binary system which, so it can occupy two states, either 0 or 1. So with $n$ bits there is $2^n$ possible states that can be represented, but only one at a time.

A qubit is a quantum state that is in a superposition of $\ket{0}$ and $\ket{1}$, so a general qubit would have the form
\begin{equation}
\label{eq:qubit}
\ket{\psi} = \alpha\ket{0} + \beta\ket{1},\,\alpha,\beta \in \mathbb{C},\, |\alpha|^2 + |\beta|^2 = 1.
\end{equation}
The qubit is no limited to 0 and 1, but can exist in a linear combination of those states. When a measurement is performed the qubit will collapse into $\ket{1}$ or $\ket{2}$ with probability $|\alpha|^2$ and $|\beta|^2$ respectively. A consequence of superposition is that $n$ qubits can represent $2^n$ states simultaneously.

The combined state of two qubits $\ket{\psi_1}$ and $\ket{\psi_2}$ is given by 
\begin{equation}
\ket{\psi_1} \otimes \ket{\psi_2} = (\alpha_1\ket{0} + \beta_1\ket{1})\otimes(\alpha_2\ket{0} + \beta_2\ket{1})
\end{equation}
the tensor product is often omitted and one would write $\ket{\psi_1} \otimes \ket{\psi_2} = \ket{\psi_1}\ket{\psi_2} = \ket{\psi_1, \psi_2}$.

Some of the common qubit gates are the Pauli gates $X,Y,Z$, the Hadamard gate $H$, and the T-gate $T$.





\subsubsection{Information stuff}

\subsubsection{Universal computation}

\subsubsection{Holonomic Quantum Computation}

