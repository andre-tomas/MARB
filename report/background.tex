\subsection{Quantum mechanics, part I}
This part offers a quick introduction to the necessary  quantum mechanics for readers whom are not familiar with the subject. The section contains nothing relevant for later parts of the thesis and can be safely skipped. For a more complete introduction I suggest chapter 1 of Sakurai.

\subsubsection{Dirac notation and quantum states}
A \textbf{quantum state} is a complex vector in a \textbf{Hilbert space}, a complete space with inner product induced metric, with finite or infinite dimension depending on characteristics of the system it represents. A quantum state vector is represented by a so-called \textbf{ket}, which is written as $\ket{\psi}$. Kets follow the same rules as a standard vector, as in that they can be added to one another, multiplied by a scalars. The ket itself is just $\ket{}$, the content of the ket is just a label which (usually) give some sense of what the state represents. \\
An \textbf{operator} is a matrix that can act on a ket, often defined by a set of rules. Written as 
\begin{equation}
\hat{O}\cdot (\ket{\psi}) = \hat{O}\ket{\psi}
\end{equation}
Since operators are matrices and kets are vectors, the existence of \textbf{eigenkets} are analogous eigenvectors. A common example of this is the time-independent Schrödinger equation
\begin{equation}
\hat{H}\ket{\psi} = E \ket{\psi}
\end{equation}
Where $\hat{H}$ is the Hamiltonian operator, and $E$ is the energy of the quantum state $\ket{\psi}$. In this case $\ket{\psi}$ is an eigenstate to the Hamiltonian.





\end{document}