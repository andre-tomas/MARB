{\footnotesize
\begin{enumerate}[label={[\arabic*]}]
\item P. W. Shor, Algorithms for quantum computation: discrete logarithms and factoring in \textit{35th Proceedings Annual Symposium on Foundations of Computer Science}, pp. 124-134 (1994).

\item L. K. Grover, Quantum Mechanics Helps in Searching for a Needle in a Haystack, Phys. Rev. Lett. \textbf{79}, 325 (1997).

\item P. Zanardi and M. Rasetti, Holonomic quantum computation, Phys. Lett. A \textbf{264}, 94 (1999).

\item E. Sjöqvist, D. M. Tong, L. M. Andersson, B. Hessmo, M. Johansson and K. Singh, Non-adiabatic holonomic quantum computation, New J. Phys. \textbf{14}, 103035 (2012).

\item M. V. Berry, Quantal phase factors accompanying adiabatic changings, Proc. Roy. Soc. London Ser. A \textbf{392}, 45 (1984).

\item Y. Aharonov and J. Anandan, Phase change during cyclic quantum evolution, Phys. Rev. Lett. \textbf{58}, 1593 (1987).

\item J. Anandan, Non-adiabatic non-abelian geometric phase, Phys. Lett. A \textbf{133}, 171 (1988).

\item F. Wilczek and A. Zee, Appearance of gauge structure in simple dynamical systems, Phys. Rev. Lett. \textbf{52}, 2111 (1984).

\item M. Glusker, D. M. Hogan and P. Vass, The ternary calculating machine of Thomas Fowler, IEEE Annals of the History of Computing \textbf{27}, 4 (2005).

\item N. P. Brusentsov and J. Ramil Alvarez, Ternary Computers: The Setun and the Setun 70, \textbf{357}, 74 (2011).

\item B. Li and Z. H. Yu and S. M. Fei, Geometry of quantum computation with qutrits, Scientific Reports \textbf{3}, 2594 (2013).

\item H. Lu, Z. Hu, M. S. Alshaykh, A. J. Moore, Y. Wang, P. Imany, A. M. Weiner and S. Kais, Quantum Phase Estimation with Time‐Frequency Qudits in a Single Photon, Advanced quantum technologies \textbf{3}, 1900074 (2020).

\item M. Luo and X. Wang, Universal quantum computation with qudits, Science China Physics, Mechanics \& Astronomy \textbf{57}, 1712 (2014).


\item Y. Wang, Z. Hu, B. C. Sanders and S. Kais, Qudits and High-Dimensional Quantum Computing, Frontiers in Physics \textbf{8}, 589504 (2020).

\item M. Z. Ai, S. Li, R. He, Z. Y. Xue, J. M. Cui, Y. F. Huang, C. F. Li and G. C. Guo, Experimental Realization of Nonadiabatic Holonomic Single-Qubit Quantum Gates with Two Dark Paths in
a Trapped Ion, accepted in Fundamental Research (2021); arXiv:2101.07483

\item J. J. Sakurai and J. Napolitano, Modern quantum mechanics; 2nd ed, San Francisco, CA: Addison-Wesley, (2011).

\item A. Pavlidis and E. Floratos, Quantum-Fourier-transform-based quantum arithmetic with qudits, Phys. Rev. A \textbf{103}, 032417 (2021).

\item M. Howard and J. Vala, Qudit versions of the qubit gate, Phys. Rev. A \textbf{86}, 022316 (2012).

\item V. O. Shkolnikov and G. Burkard, Effective Hamiltonian theory of the geometric evolution of quantum systems, Phys. Rev. A \textbf{101}, 042101 (2020).

\item M. Born and V. Fock, Beweis des Adiabatensatzes, Z. Phys. \textbf{51}, 165 (1928).

\item L.-M. Duan, J. I. Cirac and P. Zoller, Geometric Manipulation of Trapped Ion for Quantum Computations, Science \textbf{292}, 1695 (2001).

\item E. Herterich and E. Sjöqvist, Single-loop multiple-pulse nonadiabatic holonomic quantum gates, Phys. Rev. A \textbf{94}, 052310 (2016).

\item J. R. Morris and B. W. Shore, Reduction of degenerate two-level excitation to independent two-state system, Phys. Rev. A \textbf{27}, 906 (1983).

\item L. F. Shampine and M. W. Reichelt, The MATLAB ODE Suite, SIAM journal on scientific computing \textbf{18}, 1 (1997).

\item E. T. Campbell, H. Anwar and D. E. Browne, Magic-State Distillation in All Prime Dimensions Using Quantum Reed-Muller Codes, Phys. Rev. X \textbf{2}, 041021 (2012).

\end{enumerate}
}