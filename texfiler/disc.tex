%% Conclusions
We have shown how to explicitly create a quantum mechanical system, which could be used to emulate a qutrit and corresponding single-qutrit gates. This is done by expanding the dark path qubit scheme [15] into a higher dimension. We have shown how it will generalize in the qudit case and using auxiliary states to improve the robustness of the gates. Universality for the qutrit can be obtained using the Hadamard and $T$ gates in 3 dimensions. The qutrit gates have a high fidelity and their robustness is improved by the inclusion of the auxiliary state in a similar way as for the qubit [15], which suggests that this method can be beneficial for higher dimensional qudits to improve robustness. In the general case for the qudit we have also shown how any dimensional single-qudit diagonal unitary could be created by a single multi-pulse loop in parameter space and that non-diagonal unitaries scale linearly at worst in the number of loops and parameters required for control of each loop scale linearly. The possibility that the scheme expands efficiently into certain prime dimension have been discussed.
\\

%% Outlook
\noindent One of the greatest challenges of this project was how to choose parameters to replicate the unitary gates. Analytical construction of any diagonal unitary is solved, but there are many other gates that would be interesting to study in more detail. Firstly, taking a deeper look at how imposing different constraints on the parameters will impact the possible constructible gates. As well as doing it the other way around, which constraints are imposed on the parameters if one wants to construct a gate with a certain type of effect. Secondly, since some gates require more parameter space loops it raises the question of how the number of loops impacts the degrees of freedom of the final gate. If it would be possible to find approximate gate that would require fewer loops it would be  interesting to see if scheme efficiency would be more valuable than gate accuracy for some cases. This problem is very interesting since reduction of loops decreases the number of parameters needed for control, fewer loops imply less error and make the scheme less costly to run. A direct improvement to the scheme presented in this thesis would be to find a more efficient way to create the Hadamard gate, since it together with the $T$-gate constitutes a universal set. By finding an efficient implementation of the Hadamard gate the whole scheme could be improved. 

For these stated questions, the usage of numerical optimization poses interesting problems to study. Will the numerically optimized gates perform worse than analytical ones, and are there any deeper problems associated with this method? Numerical optimization in the high dimensional parameter spaces present in these system is most likely not well-behaved and may contain a lot of false minima. There might be some structure related to the parametrization that could be used to study the parameter space to gain insight into how to optimally solve the problem.
 
Another obvious concern that could be addressed is the lack of multi-qudit gates. Entanglement is one of the fundamentals of quantum computing and extending the scheme to include these would be an obvious improvement. This could maybe be achieved through some detuning or by some more explicit physical implementation similar to the qubit case in [15].

Lastly, we have shown how to construct the qudit in any dimension but only analyzed and simulated the qutrit. With the generalization in hand, writing a script which could apply the scheme to generate qudit gates of arbitrary dimension seems quite possible. Studying the larger prime dimensions to compare efficiency and check fidelity of the gates would be interesting to look into more.

Overall many aspects would be interesting to delve deeper into, and I am certain that multiple improvements to this scheme would be possible.
