\documentclass{article}
\usepackage[utf8]{inputenc}

\title{MARB: Specification}
\author{Tomas Andre \\ 
\\ 
\\ 
Supervisor: Erik Sj\"oqvist \\ 
Subject Reader: Martin Almquist}


\date{May 2021}

\begin{document}

\maketitle

\section*{Project: Holonomic optimal control for qudits}

\section*{Background} 
A circuit-based quantum computer performs efficient computation by means of 
unitary transformations (gates) acting on quantum bits (two-level systems). These 
`qubits' differ from their classical counterpart in that they can be superposed and 
entangled; features that can be used in order to design certain algorithms that are 
more efficient than what can be achieved by a classical computer \cite{shor94,grover97}. 
However, quantum computers are highly sensitive to decoherence and noise that 
may destroy the computational advantage. Therefore, it becomes pertinent to 
implement quantum gate operations that are resilient to errors. 

Holonomic control is a promising tool for implementing robust quantum gates. 
They are based on Berry phases \cite{berry84} and their non-Abelian and/or 
non-adiabatic generalizations \cite{wilczek84,aharonov87,anandan88}. These phases 
are only sensitive to the global geometry of quantum state spaces and are therefore 
conjectured to be resilient to local errors along the evolution of the quantum states. 
In particular, the non-Abelian versions of holonomic control are useful as the non-Abelian 
feature is essential as non-commuting gates are necessary for universal quantum 
computation \cite{zanardi99,sjoqvist12}. 

\section*{Description of task}
In this project, we examine holonomic optimal control \cite{liu19,li20}, which combine standard 
non-adiabatic holonomic quantum computation \cite{sjoqvist12} with inverse engineering 
and quantum control to optimize gate performance.  In particular, the focus is on extending 
holonomic optimal control  to  `qudits', which are $d\geq 3$ dimensional carriers of 
quantum information \cite{wang20}. These qudits are implemented in quantum systems, 
typically a trapped atom or ion, exhibiting a bipartite structure, defined by two coupled 
sub-groups of mutually uncoupled energy levels. One such system is the $d$-pod, in which 
$d$ uncoupled `ground state' levels, forming the qudit, is coupled to a single excited state. 
The couplings are typically induced by external laser pulses with tailored shape and duration. 

The key point with extending holonomic optimal control to qudits is two-fold: first, qudits 
allow for an exponential increase of information capacity, when the number of primitive 
systems grows; secondly, the use of qudits increases the dimensionality of the parameter space and 
thus an increased flexibility in gate optimization. The purpose of the project is to 
optimize the gate performance to systematic errors combined with decoherence 
in such qudit systems. The aim is to contribute to the realization of efficient and error 
resilient quantum gates for robust quantum computation. 

\section*{Method}
The project involves a combination of (i) analytical derivation of Hamiltonian parameters 
via inverse engineering, and (ii) numerical simulation and optimisation of qudit gates in the 
presence of systematic and decoherence-induced errors. 

The main focus is on the $d$-pod setting, which can be described by the 
Morris-Shore parametrization \cite{morris83,kyoseva06} of the corresponding state 
space. Equations for these parameters are found by a combination of inverse engineering 
of the system Hamiltonian, and a cancellation of dynamical phases yielding purely 
holonomic quantum evolution. Resilience to systematic amplitude and phase errors, as well 
as decoherence simulated by means of Lindblad-type equations \cite{nielsen00}, are 
optimized over the restricted parameter space. Specifically, the optimization is carried 
out by maximizing the fidelity of the simulated gates with respect to the ideal gates over 
the Hamiltonian parameters. 

\section*{Relevant courses}
\begin{itemize}

\item Quantum Mechanics, 1FA352

\item Quantum Information, 1FA592

\item Scientific Computing 3, 1TD397

\item Computational Physics, 1FA573

\end{itemize}

\section*{Boundaries of task}
The project is limited to: 
\begin{itemize}

\item[-] Finite dimensional Hilbert spaces with a $d$-pod structure; 

\item[-] Classical treatment of the laser fields that induce transitions between the levels; 

\item[-] Treatment of open system effects in the Markovian approximation, by means 
of the Lindblad equations. 

\end{itemize}
These limitations are designed for efficient and accurate modelling of the qudit gates.  

\section*{Time plan}
\begin{itemize}

\item[-] Week 35-36: Preparatory literature studies. 
  
\item[-] Week 36-39: Derivation of parameter equations for the inverse engineering. 

\item[-]  Week 38-45: Code development.  

\item[-] Week 43-49: Numerical optimisation of qudit gates in presence of errors. 

\item[-] Week 47-51: Writing of report. 

\item[-] Week 2 (2022): Presentation.   

\end{itemize}

\begin{thebibliography}{99}
\bibitem{shor94} P. W. Shor, 
Algorithms for quantum computation:  Discrete logarithms and  factoring, 
Proceedings  35th  annual  symposium  on  foundations  of computer science 
{\bf 1}, 124 (Murray Hill, NJ, USA, 1994). 
\bibitem{grover97} L. Grover, 
Quantum Mechanics Helps in Searching for a Needle in a Haystack, 
Phys. Rev. Lett. {\bf 79}, 325 (1997). 
\bibitem{berry84} M. V. Berry,
Quantal Phase Factors Accompanying Adiabatic Changes, 
Proc. R. Soc. London Ser. A {\bf 392}, 45 (1984).
\bibitem{wilczek84} F. Wilczek and A. Zee, 
Appearance of Gauge Structure in Simple Dynamical Systems, 
Phys. Rev. Lett. {\bf 52}, 2111 (1984). 
\bibitem{aharonov87} Y. Aharonov and J. Anandan, 
Phase change during a cyclic quantum evolution, 
Phys. Rev. Lett. {\bf 58}, 1593 (1987). 
\bibitem{anandan88} J. Anandan, 
Non-adiabatic non-Abelian geometric phase, 
Phys. Lett. A {\bf 133}, 171  (1988).
\bibitem{zanardi99} P. Zanardi and M. Rasetti, 
Holonomic quantum computation, 
Phys. Lett. A {\bf 264}, 94 (1999).
\bibitem{sjoqvist12} E. Sj\"oqvist, D. M. Tong, L. M. Andersson, B. Hessmo, 
M. Johansson, and K. Singh,  
Non-adiabatic holonomic quantum computation, 
New J. Phys. {\bf 14}, 103035 (2012).
\bibitem{wang20} Y. Wang, Z. Hu, B. C. Sanders, and S. Kais, 
Qudits and High- Dimensional Quantum Computing, 
Front. Phys. {\bf 8}, 589504 (2020).
\bibitem{liu19} B.-J. Liu, X.-K. Song, Z.-Y Xue, X. Wang, and M.-H. Yung, 
Plug-and-Play Approach to Nonadiabatic Geometric Quantum Gates, 
Phys. Rev. Lett. {\bf 123}, 100501 (2019); 
\bibitem{li20} S. Li, T. Chen, and Z.-Y. Xue, 
Fast Holonomic Quantum Computation on Superconducting Circuits With Optimal Control, 
Adv. Quantum Technol. {\bf 3}, 2000001 (2020). 
\bibitem{morris83} J. R. Morris and B. W. Shore, 
Reduction of degenerate two-level excitation to independent two-state systems, 
Phys. Rev. A {\bf 27}, 906 (1983). 
\bibitem{kyoseva06} E. S. Kyoseva and N. V. Vitanov, 
Coherent pulsed excitation of degenerate multistate systems: Exact analytic solutions, 
Phys. Rev. A {\bf 73}, 023420 (2006).
\bibitem{nielsen00} M. A. Nielsen and I. L. Chuang, 
Quantum Computation and Quantum Information, 
(Cambridge University Press, Cambridge, UK, 2000), Chap. 8.4.1.
\end{thebibliography}
\end{document}


